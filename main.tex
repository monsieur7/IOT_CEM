
        \documentclass[a4paper,12pt]{article}
        \usepackage[utf8]{inputenc}
        \usepackage[T1]{fontenc}
        \usepackage[french]{babel}
        \usepackage{hyperref}
        \usepackage{graphicx}
        \usepackage{lipsum}
        \usepackage{geometry}
        \geometry{margin=2.5cm}

        \title{Analyse des Normes CEM Applicables à un Pot Connecté}
        \author{}
        \date{}

\begin{document}

% Home Page
\begin{titlepage}
    \begin{center}
        \vspace*{2cm}
        % Align images horizontally
        \begin{minipage}{0.4\textwidth}
            \centering
            \includegraphics[width=0.8\textwidth]{imgs/polytech.png}
        \end{minipage}
        \hspace{1cm}
        \begin{minipage}{0.4\textwidth}
            \centering
            \includegraphics[width=0.8\textwidth]{imgs/ub.jpg}
        \end{minipage}

        \vspace{2cm}
        \Huge
        \textbf{Analyse des Normes CEM\\Applicables à un Pot Connecté}

        \vspace{1cm}
        \Large
        Rapport Technique

        \vspace{4cm}
        \textbf{Réalisé par :}
        \vspace{0.5cm}
        \begin{itemize}
            \item DELILLE Nolane
            \item GUER Quentin
            \item CHAPUS Loukas
            \item WENDEL Tobias
        \end{itemize}

        \vfill
        \Large
        \textbf{Décembre 2024}
    \end{center}
\end{titlepage}

\newpage
\tableofcontents
\newpage
\newpage

\section*{Introduction}
Dans un monde où les technologies connectées jouent un rôle central, le respect des normes de Compatibilité Électromagnétique (CEM) est devenu un impératif pour garantir la qualité et la sécurité des appareils électroniques.
Dans le cadre de notre projet de pot connecté, il est crucial de s'assurer que notre dispositif respecte les exigences européennes en matière de CEM.

Les normes CEM visent à prévenir les interférences nuisibles entre les appareils électroniques tout en garantissant leur bon fonctionnement dans des environnements variés.
Elles sont également essentielles pour obtenir le marquage CE, indispensable pour la commercialisation dans l'Union européenne.

Ce rapport a pour objectifs :
\begin{itemize}
    \item Identifier les principales normes applicables à notre pot connecté.
    \item Proposer des options pour réaliser les tests nécessaires, avec des solutions adaptées à différents budgets.
\end{itemize}
Ce rapport décrirat les normes adaptées pour notre projet dans le cadre d'une utilisation en Union Européenne.

\section{Normes Applicables}
Les normes applicables à notre projet sont principalement regroupées sous trois grandes directives européennes. Chaque directive couvre des aspects spécifiques de la compatibilité électromagnétique.

\subsection{Directive CEM (2014/30/UE)}
La directive CEM impose que tous les appareils électroniques respectent des limites d'émissions électromagnétiques et qu'ils soient capables de fonctionner correctement dans des environnements à fort champ électromagnétique.
Voici la liste des normes associés : % a traduire stp
\begin{enumerate}
    \item \textbf{EN 61000-6-1:2007} \\
          Generic standards - Immunity for residential, commercial, and light-industrial environments.

    \item \textbf{EN 61000-6-3:2007} \\
          Generic standards - Emission standard for residential, commercial, and light-industrial environments.

    \item \textbf{EN 50491-5-2:2010} \\
          EMC requirements for home and building electronic systems (HBES) used in residential environments.
\end{enumerate}
Voici la liste générale des tests requis par ces normes, applicable à notre cas.
\section{Tests pour la norme EN 61000-6-1:2007 (Immunité)}
\begin{enumerate}
    \item Tests d'immunité aux décharges électrostatiques.
    \item Tests d'immunité aux champs magnétiques.
    \item Tests d'immunité aux radiofréquences (modulation d'amplitude).
    \item Tests sur le port USB d'alimentation :
          \begin{enumerate}
              \item Immunité aux perturbations RF en mode commun.
              \item Immunité aux perturbations transientes rapides.
              \item Immunité aux perturbations transientes (surge).
          \end{enumerate}
\end{enumerate}
\section{Tests pour la norme EN 61000-6-3:2007 (Émissions)}
\begin{enumerate}
    \item Tests d'émissions rayonnées, sur différentes fréquences.
    \item Tests d'émissions conduites, sur différentes fréquences. (sur le port USB d'alimentation dans notre cas)
\end{enumerate}

\section{Tests pour la norme EN 50491-5-2:2010 (Compatibilité HBES/BACS)}
\begin{enumerate}
    \item \textbf{Tests d'immunité} \\
          Inclut : décharges électrostatiques, radiofréquences, champ magnétique, perturbations conduites et transitoires
          Selon chaque élemént de notre pot, il faudra réaliser les tests correspondants.
          Pour la "boite"
          \begin{enumerate}
              \item Champ magnétique
              \item radiofréquences en modulation d'amplitude
              \item décharge électrostatique
          \end{enumerate}
          Port USB :
          \begin{enumerate}
              \item Perturbations RF en mode commun
              \item Perturbations transientes rapides
              \item Perturbations transientes (surge)
          \end{enumerate}
\end{enumerate}
\subsection{Directive Radio (2014/53/UE)}
Puisque notre pot connecté utilise le Wi-Fi, il est soumis aux normes de la Directive Radio. Cette directive assure que les équipements radio respectent des exigences strictes en matière de CEM
et de performance dans les bandes de fréquences concernées.
% sortir la liste des normes svp ici 
\subsection{Directive Basse Tension (2014/35/UE)}
Cette directive s'applique aux équipements utilisant des tensions supérieures à 50V AC ou 75V DC. Toutefois, notre projet étant alimenté par USB, cette directive n'est pas directement applicable.
\section{Normes Supplémentaires}
Par contre, vu que notre pot connecté utilise une batterie lithium, d'autres normes sont applicables.
\begin{itemize}
    \item \textbf{IEC 62133 / IEC 62619} : Normes de sécurité pour les batteries rechargeables, notamment les batteries lithium-ion.
          \begin{itemize}
              \item Ces normes incluent des tests de surcharge, de protection contre les courts-circuits et de gestion des risques thermiques, en particulier pour les batteries lithium-ion, qui sont utilisés dans notre pot connecté.
          \end{itemize}
\end{itemize}
Ces normes sortent du simple cadre de la CEM, mais leur respect est essentiel pour garantir la sécurité et le bon fonctionnement de notre produit.
\subsection{Marquage CE}
Pour la commercialisation dans l'Union européenne, notre dispositif doit porter le marquage CE, qui atteste de sa conformité avec l'ensemble des directives applicables à notre produit :
\begin{itemize}
    \item Directive CEM.
    \item Directive Radio.
    \item Directives environnementales (RoHS, REACH).
\end{itemize}

\newpage

\section{Équipements de Test}
Pour garantir la conformité avec les normes CEM, il est nécessaire de réaliser un ensemble de tests Selon le budget disponible, ce rapport présente deux options pour chaque type de test : une option coûteuse, et une option low cost.

\subsection{Directive CEM}
% matcher les tests avec les normes et proposer des équipements pour les réaliser
%on peut aussi faire 1 partie test et 1 partie equipements , pour éviter les répétitions
\subsubsection{Émissions Rayonnées}
\begin{itemize}
    \item \textbf{Option coûteuse :}
          \begin{itemize}
              \item \href{https://www.keysight.com/us/en/product/N9048B/pxe-emi-receiver-1-hz-44-ghz.html}{Keysight PXE EMI Receiver} : Un récepteur EMI haute performance pour les tests d'émissions rayonnées.
              \item \href{https://eleshop.fr/gw-instek-gkt-008-emi-probe-kit.html}{Kit de sondes EMI GW Instek} : Ensemble de sondes pour une analyse précise.
          \end{itemize}
    \item \textbf{Option économique :}
          \begin{itemize}
              \item \href{https://greatscottgadgets.com/hackrf/one/}{HackRF One} : Un SDR (Software Defined Radio) polyvalent et abordable.
              \item \href{https://www.youtube.com/watch?v=2xy3Hm1_ZqI}{Antenne DIY} : Sonde simple à fabriquer en DIY.
          \end{itemize}
\end{itemize}

\subsubsection{Émissions Conduites}
\begin{itemize}
    \item \textbf{Option coûteuse :}
          \begin{itemize}
              \item \href{https://www.rohde-schwarz.com/fr/produits/test-et-mesure/tests-conduits/rs-hm6050-two-line-v-network-lisn_63493-48135.html}{Rohde \& Schwarz LISN} : Réseau de stabilisation d'impédance pour les tests précis.
                    % surrement d'autres trucs ici genre un outil de mesure de courant / tension a voir 
          \end{itemize}
    \item \textbf{Option économique :}
          \begin{itemize}
              \item \href{https://hackaday.io/project/181265-diy-cispr-25-lisn}{LISN DIY} : Construction d’un LISN à faible coût.
              \item % same thing here, a voir pour un outil de mesure de courant / tension
          \end{itemize}
\end{itemize}

\subsubsection{Immunité}
\begin{itemize}
    \item \textbf{Option coûteuse :}
          \begin{itemize}
              \item \href{https://www.keysight.com/us/en/assets/7018-05702/technical-overviews/5992-2241.pdf}{Générateur de signal Rohde \& Schwarz}
              \item \href{}{} % antenne pour les tests d'immunité
              \item \href{https://www.schloeder-emc.com/emc-products/emc-test-and-measurement-system/esd-simulators/esd-simulator-165-kv.html}{Genérateur ESD} % generateur ESD
              \item \href{}{} % transient generator (voltage surge)
                    %https://www.emc-directory.com/community/what-is-cispr-35_14
                    %https://www.perplexity.ai/search/with-the-emc-directive-what-te-C87OInrkSdGRJ9UkkdSjQA
          \end{itemize}
    \item \textbf{Option économique :}
          \begin{itemize}
              \item \href{https://www.crowdsupply.com/era-instruments/erasynth-micro}{ESD Gun DIY} : Générateur ESD à faible coût.
              \item \href{}{} %gen rf pas cher / hackrf ? 
              \item \href{}{} %antenne DIY
              \item \href{}{} % transient generator DIY
          \end{itemize}
\end{itemize}

\newpage

\subsection{Directive RED}
Pour chaque norme, voici les tests requis et les équipements recommandés pour les réaliser.

\subsubsection{EN 301 489-1 : Exigences Générales de CEM} % a verifer si c'est bien la norme pour les tests de CEM dans la norme RED 
\subsubsection*{Tests Requis :}
\begin{itemize}
    \item \textbf{Émissions rayonnées} : Vérifie que les émissions électromagnétiques restent dans les limites autorisées.
    \item \textbf{Immunité rayonnée} : Vérifie la résistance aux interférences électromagnétiques.
    \item \textbf{Émissions conduites} : Vérifie les perturbations transmises par les câbles, quelles qu'ils soient, restent dans les limites autorisées.
\end{itemize}

\subsubsection*{Équipements :}
\begin{itemize}
    \item \textbf{Option coûteuse :}
          \begin{itemize}
              \item Récepteur EMI Keysight PXE (\href{https://www.keysight.com/us/en/product/N9048B/pxe-emi-receiver-1-hz-44-ghz.html}{N9048B}).
              \item Antennes EMI GW Instek (\href{https://eleshop.fr/gw-instek-gkt-008-emi-probe-kit.html}{Kit GKT-008}).
          \end{itemize}
    \item \textbf{Option économique :}
          \begin{itemize}
              \item HackRF One (\href{https://greatscottgadgets.com/hackrf/one/}{Lien officiel}).
              \item Antenne DIY (\href{https://www.youtube.com/watch?v=2xy3Hm1_ZqI}{Tutoriel YouTube}).
          \end{itemize}
\end{itemize}

\subsubsection{EN 301 489-17 : Exigences pour Wi-Fi et Bluetooth}
\subsubsection*{Tests Requis :}
\begin{itemize}
    \item \textbf{Tests d'Immunité} : Vérifie le fonctionnement d'un fort champ électromagnétique.
    \item \textbf{Émissions parasites} : Vérifie que l’appareil n’émet pas en dehors de sa bande allouée.
\end{itemize}

\subsubsection*{Équipements :}
\begin{itemize}
    \item \textbf{Option coûteuse :}
          \begin{itemize}
              \item Analyseur de spectre Rohde \& Schwarz FSW (\href{https://www.rohde-schwarz.com/product/FSW.html}{FSW}).
              \item %antennes à mettre
          \end{itemize}
    \item \textbf{Option économique :}
          \begin{itemize}
              \item HackRF One (\href{https://greatscottgadgets.com/hackrf/one/}{Lien officiel}).
              \item % antennes à mettre aussi 
          \end{itemize}
\end{itemize}

\subsubsection{ETSI EN 300 328 : Appareils Wi-Fi 2,4 GHz}
\subsubsection*{Tests Requis :}
\begin{itemize}
    \item \textbf{Puissance d’émission} : Vérifie que la puissance reste dans les limites légales.
    \item \textbf{Stabilité de fréquence} : Vérifie que le signal reste dans la bande autorisée.
    \item \textbf{Taux de fuite dans les bandes adjacentes (ACLR)} : Vérifie les interférences dans les bandes voisines.
    \item \textbf{Sensibilité du récepteur} : Vérifie le niveau minimal de signal requis pour un fonctionnement correct.
\end{itemize}

\subsubsection*{Équipements :}
\begin{itemize}
    \item \textbf{Option coûteuse :}
          \begin{itemize}
              \item Analyseur de spectre Rohde \& Schwarz FSV3000 (\href{https://www.rohde-schwarz.com/product/FSV3000.html}{FSV3000}).
              \item Wattmetre RF, série U2040 (\href{https://www.keysight.com/us/en/assets/7018-04521/data-sheets/5992-0040.pdf}{Lien Officiel}).
          \end{itemize}
    \item \textbf{Option économique :}
          \begin{itemize}
              \item HackRF One (\href{https://greatscottgadgets.com/hackrf/one/}{Lien officiel}).
              \item Détecteur de puissance RF DIY (\href{https://www.motorobit.com/usb-rf-power-meter-v30-100k-10ghz-rf-power-meter-with-display}{Elektor RF Power Meter}).
          \end{itemize}
\end{itemize}

\section{Configuration Recommandée pour les Tests}
\begin{itemize}
    \item \textbf{Chambre blindée ou anéchoïque}:
          \begin{itemize}
              \item \textbf{Option coûteuse :} Chambres préfabriquées (ex. TDK ou ETS Lindgren). % a verifier + liens 
              \item \textbf{Option économique :} Enceinte blindée DIY (ex. maillage en cuivre ou mousse absorbante RF). % mettre lien
          \end{itemize}
    \item \textbf{Outils logiciels}:
          \begin{itemize}
              \item GNU Radio ou SDR\# pour les configurations économiques.
              \item Logiciels propriétaires pour les équipements coûteux. % liens vers les logiciels
          \end{itemize}
\end{itemize}

\section*{Conclusion}
La conformité CEM est essentielle pour garantir la qualité et la sécurité de notre pot connecté. En respectant les normes EN 55032 et EN 55035, et en utilisant des équipements adaptés, nous pouvons répondre aux exigences réglementaires tout en optimisant nos ressources. Ce rapport fournit une base solide pour aborder les tests de conformité et préparer la mise en marché de notre produit.

\end{document}
\end{itemize}

\section*{Conclusion}
La conformité CEM est essentielle pour garantir la qualité et la sécurité de notre pot connecté. En respectant les normes EN 55032 et EN 55035, et en utilisant des équipements adaptés, nous pouvons répondre aux exigences réglementaires tout en optimisant nos ressources. Ce rapport fournit une base solide pour aborder les tests de conformité et préparer la mise en marché de notre produit.
%a refaire pour la conclusion en y ajoutant les autres normes 
\end{document}
