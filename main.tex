
        \documentclass[a4paper,12pt]{article}
        \usepackage[utf8]{inputenc}
        \usepackage[T1]{fontenc}
        \usepackage[french]{babel}
        \usepackage{hyperref}
        \usepackage{graphicx}
        \usepackage{lipsum}
        \usepackage{geometry}
        \geometry{margin=2.5cm}

        \title{Analyse des Normes CEM Applicables à un Pot Connecté}
        \author{}
        \date{}

\begin{document}

% Home Page
\begin{titlepage}
    \begin{center}
        \vspace*{2cm}
        % Align images horizontally
        \begin{minipage}{0.4\textwidth}
            \centering
            \includegraphics[width=0.8\textwidth]{imgs/polytech.png}
        \end{minipage}
        \hspace{1cm}
        \begin{minipage}{0.4\textwidth}
            \centering
            \includegraphics[width=0.8\textwidth]{imgs/ub.jpg}
        \end{minipage}

        \vspace{2cm}
        \Huge
        \textbf{Analyse des Normes CEM\\Applicables à un Pot Connecté}

        \vspace{1cm}
        \Large
        Rapport Technique

        \vspace{4cm}
        \textbf{Réalisé par :}
        \vspace{0.5cm}
        \begin{itemize}
            \item DELILLE Nolane
            \item GUER Quentin
            \item CHAPUS Loukas
            \item WENDEL Tobias
        \end{itemize}

        \vfill
        \Large
        \textbf{Décembre 2024}
    \end{center}
\end{titlepage}

\newpage
\tableofcontents
\newpage
\newpage

\section*{Introduction}
Dans un monde où les technologies connectées jouent un rôle central, le respect des normes de Compatibilité Électromagnétique (CEM) est devenu un impératif pour garantir la qualité et la sécurité des appareils électroniques. Dans le cadre de notre projet de pot connecté, il est crucial de s'assurer que notre dispositif respecte les exigences européennes en matière de CEM.

Les normes CEM visent à prévenir les interférences nuisibles entre les appareils électroniques tout en garantissant leur bon fonctionnement dans des environnements variés. Elles sont également essentielles pour obtenir le marquage CE, indispensable pour la commercialisation dans l'Union européenne.

Ce rapport a pour objectifs :
\begin{itemize}
    \item Identifier les principales normes applicables à notre pot connecté.
    \item Proposer des options pour réaliser les tests nécessaires, avec des solutions adaptées à différents budgets.
\end{itemize}
Ce rapport décrirat les normes adaptées pour notre projet dans le cadre d'une utilisation en Union Européenne.

\section{Normes Applicables}
Les normes applicables à notre projet sont principalement regroupées sous trois grandes directives européennes. Chaque directive couvre des aspects spécifiques de la compatibilité électromagnétique ou des exigences techniques.

\subsection{Directive CEM (2014/30/UE)}
La directive CEM impose que tous les appareils électroniques respectent des limites d'émissions électromagnétiques et qu'ils soient capables de fonctionner correctement dans des environnements perturbés. Les normes principales associées à cette directive sont :
\begin{itemize}
    \item \textbf{EN 55032 (CISPR 32)} : Elle encadre les émissions électromagnétiques des équipements multimédias afin de limiter les interférences avec d'autres appareils.
    \item \textbf{EN 55035 (CISPR 35)} : Elle garantit l'immunité des appareils multimédias face aux perturbations électromagnétiques externes.
\end{itemize}

\subsection{Directive Radio (2014/53/UE)}
Puisque notre pot connecté utilise le Wi-Fi, il est soumis aux normes de la Directive Radio. Cette directive assure que les équipements radio respectent des exigences strictes en matière de CEM et de performance dans les bandes de fréquences concernées :
\begin{itemize}
    \item \textbf{EN 301 489-1} : Norme générale sur la CEM pour les équipements radio.
    \item \textbf{EN 301 489-17} : Norme spécifique aux dispositifs Wi-Fi et Bluetooth.
    \item \textbf{ETSI EN 300 328} : Exigences techniques pour les appareils Wi-Fi opérant dans la bande ISM 2,4 GHz.
\end{itemize}

\subsection{Directive Basse Tension (2014/35/UE)}
Cette directive s'applique aux équipements utilisant des tensions supérieures à 50V AC ou 75V DC. Toutefois, notre projet étant alimenté par USB, cette directive n'est pas directement applicable.
\section{Normes Supplémentaires}
\begin{itemize}
    \item \textbf{EN 61000-6-1} : Exigences générales d'immunité pour les environnements résidentiels, commerciaux et industriels légers.
    \item \textbf{EN 61000-6-3} : Limites d'émissions pour ces mêmes environnements.
    \item \textbf{IEC 62133 / IEC 62619} : Normes de sécurité pour les batteries rechargeables, notamment les batteries lithium-ion.
          \begin{itemize}
              \item Ces normes incluent des tests de surcharge, de protection contre les courts-circuits et de gestion des risques thermiques, en particulier pour les batteries lithium-ion, qui sont utilisés dans notre pot connecté.
          \end{itemize}
\end{itemize}
\subsection{Marquage CE}
Pour la commercialisation dans l'Union européenne, notre dispositif doit porter le marquage CE, qui atteste de sa conformité avec les directives :
\begin{itemize}
    \item Directive CEM.
    \item Directive Radio.
    \item Toute autre directive applicable, telle que RoHS pour la limitation des substances dangereuses, par exemple le plomb dans les soudures.
\end{itemize}

\newpage

\section{Équipements de Test}
Pour garantir la conformité avec les normes CEM, il est nécessaire de réaliser des tests d’émissions et d’immunité. Selon le budget disponible, voici deux options pour chaque type de test : une option coûteuse pour les laboratoires ou entreprises équipés, et une option économique pour les petits projets ou prototypes.

\subsection{Directive CEM}
\subsubsection{Émissions Rayonnées}
\begin{itemize}
    \item \textbf{Option coûteuse :}
          \begin{itemize}
              \item \href{https://www.keysight.com/us/en/product/N9048B/pxe-emi-receiver-1-hz-44-ghz.html}{Keysight PXE EMI Receiver} : Un récepteur EMI haute performance pour les tests d'émissions rayonnées.
              \item \href{https://eleshop.fr/gw-instek-gkt-008-emi-probe-kit.html}{Kit de sondes EMI GW Instek} : Ensemble de sondes pour une analyse précise.
          \end{itemize}
    \item \textbf{Option économique :}
          \begin{itemize}
              \item \href{https://greatscottgadgets.com/hackrf/one/}{HackRF One} : Un SDR polyvalent pour une analyse simple des émissions.
              \item \href{https://www.youtube.com/watch?v=2xy3Hm1_ZqI}{Antenne DIY} : Un tutoriel pour créer une antenne à faible coût.
          \end{itemize}
\end{itemize}

\subsubsection{Émissions Conduites}
\begin{itemize}
    \item \textbf{Option coûteuse :}
          \begin{itemize}
              \item \href{https://www.rohde-schwarz.com/fr/produits/test-et-mesure/tests-conduits/rs-hm6050-two-line-v-network-lisn_63493-48135.html}{Rohde \& Schwarz LISN} : Réseau de stabilisation d'impédance pour les tests précis.
          \end{itemize}
    \item \textbf{Option économique :}
          \begin{itemize}
              \item \href{https://hackaday.io/project/181265-diy-cispr-25-lisn}{LISN DIY} : Construction d’un LISN avec GNU Radio.
          \end{itemize}
\end{itemize}

\subsubsection{Immunité}
\begin{itemize}
    \item \textbf{Option coûteuse :}
          \begin{itemize}
              \item \href{https://www.keysight.com/us/en/assets/7018-05702/technical-overviews/5992-2241.pdf}{Générateur de signal Rohde \& Schwarz} : Outil avancé pour les tests d’immunité.
          \end{itemize}
    \item \textbf{Option économique :}
          \begin{itemize}
              \item \href{https://www.crowdsupply.com/era-instruments/erasynth-micro}{ESD Gun DIY} : Générateur ESD à faible coût.
          \end{itemize}
\end{itemize}

\newpage

\subsection{Directive RED}
Pour chaque norme, voici les tests requis et les équipements recommandés pour les réaliser.

\subsubsection{EN 301 489-1 : Exigences Générales de CEM}
\subsubsection*{Tests Requis :}
\begin{itemize}
    \item \textbf{Émissions rayonnées} : Vérifie les émissions dans le spectre radio.
    \item \textbf{Immunité rayonnée} : Vérifie la résistance aux interférences électromagnétiques.
    \item \textbf{Émissions conduites} : Vérifie les perturbations transmises par les câbles.
\end{itemize}

\subsubsection*{Équipements :}
\begin{itemize}
    \item \textbf{Option coûteuse :}
          \begin{itemize}
              \item Récepteur EMI Keysight PXE (\href{https://www.keysight.com/us/en/product/N9048B/pxe-emi-receiver-1-hz-44-ghz.html}{N9048B}).
              \item Antennes EMI GW Instek (\href{https://eleshop.fr/gw-instek-gkt-008-emi-probe-kit.html}{Kit GKT-008}).
          \end{itemize}
    \item \textbf{Option économique :}
          \begin{itemize}
              \item HackRF One (\href{https://greatscottgadgets.com/hackrf/one/}{Lien officiel}).
              \item Antenne DIY (\href{https://www.youtube.com/watch?v=2xy3Hm1_ZqI}{Tutoriel YouTube}).
          \end{itemize}
\end{itemize}

\subsubsection{EN 301 489-17 : Exigences pour Wi-Fi et Bluetooth}
\subsubsection*{Tests Requis :}
\begin{itemize}
    \item \textbf{Brouillage du récepteur} : Vérifie le fonctionnement en présence de signaux forts.
    \item \textbf{Émissions parasites} : Vérifie que l’appareil n’émet pas en dehors de sa bande allouée.
\end{itemize}

\subsubsection*{Équipements :}
\begin{itemize}
    \item \textbf{Option coûteuse :}
          \begin{itemize}
              \item Analyseur de spectre Rohde \& Schwarz FSW (\href{https://www.rohde-schwarz.com/product/FSW.html}{FSW}).
          \end{itemize}
    \item \textbf{Option économique :}
          \begin{itemize}
              \item RTL-SDR (\href{https://www.rtl-sdr.com/}{Lien officiel}).
          \end{itemize}
\end{itemize}

\subsubsection{ETSI EN 300 328 : Appareils Wi-Fi 2,4 GHz}
\subsubsection*{Tests Requis :}
\begin{itemize}
    \item \textbf{Puissance d’émission} : Vérifie que la puissance reste dans les limites légales.
    \item \textbf{Stabilité de fréquence} : Vérifie que le signal reste dans la bande autorisée.
    \item \textbf{Taux de fuite dans les bandes adjacentes (ACLR)} : Vérifie les interférences dans les bandes voisines.
    \item \textbf{Sensibilité du récepteur} : Vérifie le niveau minimal de signal requis pour un fonctionnement correct.
\end{itemize}

\subsubsection*{Équipements :}
\begin{itemize}
    \item \textbf{Option coûteuse :}
          \begin{itemize}
              \item Analyseur de spectre Rohde \& Schwarz FSV3000 (\href{https://www.rohde-schwarz.com/product/FSV3000.html}{FSV3000}).
              \item Wattmètre Keysight N1914A (\href{https://www.keysight.com/us/en/product/N1914A.html}{Lien officiel}).
          \end{itemize}
    \item \textbf{Option économique :}
          \begin{itemize}
              \item HackRF One (\href{https://greatscottgadgets.com/hackrf/one/}{Lien officiel}).
              \item Détecteur de puissance RF DIY (\href{https://www.elektor.com/rf-power-meter}{Elektor RF Power Meter}).
          \end{itemize}
\end{itemize}

\section{Configuration Recommandée pour les Tests}
\begin{itemize}
    \item \textbf{Chambre blindée ou anéchoïque}:
          \begin{itemize}
              \item \textbf{Option coûteuse :} Chambres préfabriquées (ex. TDK ou ETS Lindgren).
              \item \textbf{Option économique :} Enceinte blindée DIY (ex. maillage en cuivre ou mousse absorbante RF).
          \end{itemize}
    \item \textbf{Outils logiciels}:
          \begin{itemize}
              \item GNU Radio ou SDR\# pour les configurations économiques.
              \item Logiciels propriétaires pour les équipements coûteux.
          \end{itemize}
\end{itemize}

\section*{Conclusion}
La conformité CEM est essentielle pour garantir la qualité et la sécurité de notre pot connecté. En respectant les normes EN 55032 et EN 55035, et en utilisant des équipements adaptés, nous pouvons répondre aux exigences réglementaires tout en optimisant nos ressources. Ce rapport fournit une base solide pour aborder les tests de conformité et préparer la mise en marché de notre produit.

\end{document}
\end{itemize}

\section*{Conclusion}
La conformité CEM est essentielle pour garantir la qualité et la sécurité de notre pot connecté. En respectant les normes EN 55032 et EN 55035, et en utilisant des équipements adaptés, nous pouvons répondre aux exigences réglementaires tout en optimisant nos ressources. Ce rapport fournit une base solide pour aborder les tests de conformité et préparer la mise en marché de notre produit.
%a refaire pour la conclusion en y ajoutant les autres normes 
\end{document}
